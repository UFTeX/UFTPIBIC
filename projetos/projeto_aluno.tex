\documentclass[12pt,consuni]{uftpibic}

\usepackage[alf,abnt-emphasize=bf]{../abntex2/abntex2cite}
%\renewcommand{\backrefpagesname}{}
%\renewcommand{\backref}{}
\renewcommand*{\backrefalt}[4]{}
%\usepackage{cite}
\usepackage{tikz}
\usepackage{multirow}
\graphicspath{{}{images/}{figures/}} 
%-------------------------------------------------------------------------------------------------------------------------%
% Endereço Institucional
%-------------------------------------------------------------------------------------------------------------------------%
\address{Quadra 109 Norte, Av. Ns 15, ALCNO 14, Prédio da Reitoria, Proesq}
\cep{77001-090}
\phone{(63) 3229-4037}
\mail{propesq@uft.edu.br}
\city{Palmas}
%-------------------------------------------------------------------------------------------------------------------------%
% Dados do Projeto
%-------------------------------------------------------------------------------------------------------------------------%
\title{Implementação e testes de eficiência energética de circuito de soma em FPGA}

\advisor{Prof.}{Tiago da Silva}{Almeida}{Me.}
\author{João Pedro Oliveira}{Barbosa}
\department{Ciência da Computação}

\keyword{Eficiência Energética}
\keyword{FPGA}
\keyword{Somador}
\keyword{Verilog}
\keyword{Desempenho}
\keyword{Vivado}

%-------------------------------------------------------------------------------------------------------------------------%
% Para evitar a quebra automática dos Capítulos, descomentar abaixo
%-------------------------------------------------------------------------------------------------------------------------%
\makeatletter
\patchcmd{\chapter}{\if@openright\cleardoublepage\else\clearpage\fi}{}{}{}
\makeatother
%-------------------------------------------------------------------------------------------------------------------------%
\begin{document}

\maketitle

%\onehalfspacing
%\pagebreak
\vspace{1cm}
\setlength{\parindent}{1.5cm}
\chapter{Introdução}

\textit{Field Programmable Gate Array} (FPGA) é um tipo de circuito integrado versátil projetado para ser programável (e muitas vezes reprogramável), usado para atender a diferentes propósitos, principalmente computação de alto desempenho e prototipagem \cite{ibm_fpga_2024}. É um dispositivo amplamente utilizado devido à sua versatilidade e poder de processamento e atualmente pode ser encontrado em segmentos automotivo, aeroespacial, industrial, telecomunicações e vários outros.

Este dispositivo é composto por \textit{programmable logic blocks} (PLBs) que implementam as funções lógicas, \textit{programmable routing} que conectam essas funções lógicas e blocos de entrada/saída que são conectados aos blocos lógicos por meio das interconexões de roteamento e que fazem conexões com dispositivos externos (\textit{off-chip}) \cite{Farooq2012}. Graças à sua arquitetura, é possível realizar vários testes e aplicações, como execução de algoritmos para análise de desempenho até operações em paralelo que podem ser mais rápidas do que em processadores (CPU - \textit{Central Processing Unit}) convencionais.

Em qualquer sistema computacional a operação mais fundamental é a operação de soma, que são realizadas por circuitos lógicos chamados somadores. Estes circuitos são tão elementares que são utilizados na construção de outros circuitos aritméticos (CA), tais quais subtratores, multiplicadores e divisores. Embora seja simples, é a base para todas as operações matemáticas fundamentais, é o componente central da Unidade Lógica e Aritmética (ULA), que é responsável pelas operações matemáticas do processador \cite{toccisistemas}, e é aplicado em contadores e endereçamento de memória. Como é um circuito utilizado com muita frequência, profissionais investem em sua otimização. Ou seja, o desempenho de um processador está diretamente ligado à eficiência do somador.

\vspace{1.5cm}
\chapter{Objetivos}

O objetivo deste trabalho é contribuir com pesquisa e testes em placa FPGA para a análise de eficiência energética de diferentes arquiteturas de somadores digitais, e servir como material de referência e base de estudos para outros alunos que forem realizar pesquisas, projetos e testes na área, fornecendo informações e resultados que os auxiliem na execução de suas tarefas. Os passos se darão da seguinte maneira:

\begin{itemize}
\item Selecionar e estudar os principais tipos de somadores digitais;
\item Implementação dos somadores em Verilog;
\item Mapeamento em um FPGA da Xilinx;
\item Comparação considerando área, potência e frequência;
%\item Aplicação de técnicas de otimização, como \textit{clock gating}, para redução de consumo.
\item Realização de medições de consumo energético, desempenho e temperatura.
\end{itemize}


\vspace{1.5cm}
\chapter{Plano de trabalho individualizado}

O projeto será realizado a partir de pesquisas e implementações em uma placa \textit{Boolean Board}. Inicialmente, será necessário o estudo teórico mais profundo sobre somadores, que consistem em circuitos lógicos que fazem a soma de dois números binários e geralmente podem ser divididos entre meio somador (\textit{Half-Adder} - HA) e somador completo (\textit{Full-Adder} - FA), que são os principais tipos de somadores, para depois iniciar os testes na placa em si \cite{toccisistemas}. 

A atividade de projeto em FPGAs envolve alguns passos básicos: i) análise de código em HDL (\textit{Hardware Description Language}), ii) síntese, que é a tradução do projeto em relação à estrutura de PLBs; iii) mapeamento da quantidade de PLBs necessária ao projeto e o que existe disponível na FPGA; iv) e por fim o roteamento dessa estrutura para o modelo de FPGA específico e geração do \textit{stream} de bits para gravação na FPGA. 

Ainda, HDL é uma linguagem de programação que tem por função especificar o comportamento, a estrutura e o funcionamento de um circuito digital e possui características particulares como expressar \textit{clocks}, sequenciar operações e descrever construções usando vetores \cite{Flake2020}. 

Os códigos HDL serão realizados no software Vivado e, então, após analisar e concluir a implementação dos somadores. O ambiente de projeto do Vivado possui um analisador de consumo de energia baseado em arquivos tipo SAIF (\textit{Switching Activity Interchange Format}) que serão usados nessa pesquisa, além de sensores já embutidos na placa de projeto. 

\vspace{1.5cm}
\chapter{Cronograma de execução}

As atividades do trabalho são divididas conforme a Tabela \ref{tb:atividades}, enumeradas de A à H, e sua execução é de acordo com o cronograma da Tabela \ref{tb:cronograma}.

\begin{table}[!h]
  \centering
  %\footnotesize
  \caption{Lista de atividades previstas.}\label{tb:atividades}
  \begin{tabular}{cp{9.4cm}}
    \toprule
    {\bf Atividades} & \multicolumn{1}{c}{\bf Descrição} \\
    \midrule
    &\\[-0.4cm]
    \textbf{A} &  Estudo teórico sobre FPGAs, somadores e Verilog. \\[0.2cm]
    \textbf{B} &  Implementação de diferentes arquiteturas de somadores em Verilog.\\[0.2cm]
    \textbf{C} &  Mapear o FPGA utilizado para os testes.\\[0.2cm]
    \textbf{D} &  Testes práticos e simulações no Vivado.\\[0.2cm]
    \textbf{E} &  Coleta e análise de métricas (área, potência, frequência). \\[0.2cm]
    \textbf{F} &  Aplicação de técnicas de otimização. \\[0.2cm]
    \textbf{G} &  Escrita de relatório parcial em forma de trabalhos parciais para publicação em congressos e/ou periódicos e como relatório de prestação de contas com os sistemas de gestão da UFT. \\[0.2cm]
    \textbf{H} &  Escrita de relatório final em forma de trabalhos parciais para publicação em congressos e/ou periódicos e como relatório de prestação de contas com os sistemas de gestão da UFT. \\[0.2cm]
    \bottomrule
  \end{tabular}
\end{table}


\begin{table}[!ht]
  \centering %\fontsize{8}{12}%\tiny
  \caption{Cronograma previsto das atividades descritas na Tabela \ref{tb:atividades}, destacando sua execução mês a mês.}\label{tb:cronograma}
  \begin{tabular}{|c|c|c|c|c|c|c|c|c|c|c|c|c|}
    \hline
    Ano  &\multicolumn{12}{c|}{2025/2026}\\
    \hline
 {Mês} &
\multirow{2}*{Set}&\multirow{2}*{Out}&\multirow{2}*{Nov}&\multirow{2}*{Dez}&\multirow{2}*{Jan}&\multirow{2}*{Fev}&
\multirow{2}*{Mar}&\multirow{2}*{Abr}&\multirow{2}*{Mai}&\multirow{2}*{Jun}&\multirow{2}*{Jul}&\multirow{2}*{Ago} \\
   \cline{1-1}
Atv. & & & & & & & & & & & &   \\
\hline
A & X & X & X & X &   &   &   &   &   &   &   &   \\
\hline
B &   &   &   & X & X & X &   &   &   &   &   &   \\
\hline
C &   &   &   &   & X & X & X &   &   &   &   &   \\
\hline
D &   &   &   &   &   & X & X & X & X & X &   &   \\
\hline
E &   &   &   &   &   & X & X & X & X & X &   &   \\
\hline
F &   &   &   &   &   &   &   &   & X & X & X &   \\
\hline
G &   &   &   & X &   &   &   &   &   &   &   &   \\
\hline
H &   &   &   &   &   &   &   &   &   &   &   & X \\
\hline
  \end{tabular}
\end{table}

\vspace{1.25cm}
\bibliography{bibliografia}

\end{document}