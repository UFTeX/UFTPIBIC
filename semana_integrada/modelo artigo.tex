\documentclass[12pt]{uftpibicsic}

\usepackage[alf,abnt-emphasize=bf]{abntex2cite}
\renewcommand{\backrefpagesname}{}
\renewcommand{\backref}{}
\renewcommand*{\backrefalt}[4]{}
\renewcommand{\bibname}{Literatura Citada}

\graphicspath{{figures/} {imagens/} {}}

%-------------------------------------------------------------------------------------------------------------------------%
% Dados do Projeto
%-------------------------------------------------------------------------------------------------------------------------%

\title{RUTI - Rastreamento Urbano de Transporte Integrado: módulo de comunicação e correção de erros na estação}

\author{Leonardo Rezende Costa\inst{1} \\Prof Me. Tiago da Silva Almeida\inst{2}}

\address{Aluno do Curso de Ciência da Computação;  Campus de Palmas; e-mail: leonardorec1@gmail.com; PIVIC/UFT;
\nextinstitute
Orientador do Curso de Ciência da Computação; Campus de Palmas; e-mail: tiagoalmeida@uft.edu.br}

\campus{Câmpus de Palmas}
\area{Ciências Exatas e da Terra}

  % ---- Palavras-chaves em português do trabalho
  \keyword{FPGA}
  \keyword{Multiplicadores}
  \keyword{Computação Aproximada}
  \keyword{HDL}
  \keyword{Eficiência Energética}
  % ---- Palavras-chaves em inglês do trabalho
  \foreignkeyword{FPGA}
  \foreignkeyword{Multipliers}
  \foreignkeyword{Approximate Computing}
  \foreignkeyword{HDL}
  \foreignkeyword{Energy Efficiency}



\begin{document}

\mainmatter
\maketitle

\begin{spacing}{1}
\begin{abstract}

\end{abstract}

\begin{foreignabstract}

\end{foreignabstract}
\end{spacing}

%%%%%%%%%%%%%%%%%%%%%%%%%%%%%%%%%%%%%%%%%%%%%%%%%%%%%%%%%%%%%%%%%%%%%%%%%%%%%%%%%%%%%%

\chapter{Introdução}
% Nesse ponto é necessário dar um contexto sobre o trabalho. Pense em um texto que responde as seguintes perguntas:

% O que é eficiência energética?

% Por que é importante estudar esse assunto?

% Qual é o problema que você vai estudar?

% Após explicar sobre isso é o momento de dizer qual é a sua pergunta de pesquisa, por exemplo:

\begin{quote}
Qual a diferença no consumo de energia de diferentes divisores aproximados em dispositivos FPGA e como esse consumo de energia se relaciona com o erro do divisor?
\end{quote}

% Aqui é importante explicar a pergunta, ou relaciona-la com o problema e área de pesquisa que foi discutida acima


% Após vem as hipóteses de pesquisa, por exemplo:

Hipótese nula ($H_{1_{0}}$) é \textit{todos os modelos de somadores aproximados tem o mesmo resultado para todas as variáveis dependentes $(\mu_{(1)}, \mu_{(2)}, ..., \mu_{(n)})$ (energia e erro)}. A hipótese alternativa ($H_{1_{a}}$) é \textit{há uma diferença entre somadores aproximados diferentes para pelo menos um par $i$, $j$ para $i \neq j$, sendo $i \neq j$}.

% Aqui você pode explicar melhor e definir os objetivos do estudo, como:

Analisar \rule{5cm}{0.4pt}
%<Object(s) of study>
com o objetivo de \rule{5cm}{0.4pt}
%<Purpose>
com relação à \rule{5cm}{0.4pt}
%<Quality focus>
do ponto de vista do \rule{5cm}{0.4pt}
%<Perspective>
no contexto de \rule{5cm}{0.4pt}
%<Context>.

\chapter{Trabalhos relacionados e fundamentos}
% Nesse ponto do texto, você pode explicar os 5 artigo que você selecionou sobre a literatura. (não esquece de colocar a string de busca e falar qual a base de dados você usou)
% Além disso, imagine que o leitor não sabe nada sobre o assunto, então você vai explicar:

% O que é um somador? (exemplo)

% O que é um multiplicador? (exemplo)

% O que é computação approximada? (exemplo)

% O que é FPGA? (exemplo)

% Como o seu trabalho está relacionado com os 5 que você explicou? (exemplo)


\chapter{Materiais, métodos e resultados esperados}
% Aqui você vai explicar como você vai fazer seu trabalho, por exemplo, mostrar:

% As métricas de erro que você vai usar
% Os softwares e linguagem que você vai usar
% Qual modelo de FPGA você vai usar
% Como você vai capturar os resultados (simulação ou leitura da placa)
% Como você vai testar a hipótese. Aqui você pode descrever e usar os testes de hipótese clássicos, como: regressão linear, ANOVA e t-test.

\chapter{Considerações finais e cronograma de trabalho}
% Aqui é só recaptular tudo que foi dito e fazer um cronograma do que ainda falta ser feito.

% Minha sugestão é explicar de maneira clara o que ainda será feito e criar um cronograma para o relatório final.

\bibliography{bibliografia}


\end{document}

